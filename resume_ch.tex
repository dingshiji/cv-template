%%%%%%%%%%%%%%%%%%%%%%%%%%%%%%%%%%%%%%%%%
% Medium Length Professional CV
% LaTeX Template
% Version 2.0 (8/5/13)
%
% This template has been downloaded from:
% http://www.LaTeXTemplates.com
%
% Original author:
% Trey Hunner (http://www.treyhunner.com/)
%
% Important note:
% This template requires the resume.cls file to be in the same directory as the
% .tex file. The resume.cls file provides the resume style used for structuring the
% document.
%
%%%%%%%%%%%%%%%%%%%%%%%%%%%%%%%%%%%%%%%%%

%----------------------------------------------------------------------------------------
%	PACKAGES AND OTHER DOCUMENT CONFIGURATIONS
%----------------------------------------------------------------------------------------

\documentclass{resume} % Use the custom resume.cls style
\usepackage[left=0.75in,top=0.6in,right=0.75in,bottom=0.6in]{geometry} % Document margins
\usepackage{xeCJK}  
\usepackage{color}
\usepackage[citecolor=green
            ]{hyperref}

\name{丁世纪} % Your name
\address{+(86) 13310003072 \\ 181860014@smail.nju.edu.cn} % Your phone number and email
\address{江苏省南京市仙林大道163号,邮编210046}  % Your address
\begin{document}

%----------------------------------------------------------------------------------------
%	EDUCATION SECTION
%----------------------------------------------------------------------------------------

\begin{rSection}{教育}
{\textbf{南京大学}} \hfill {\em 2018.9 - 至今} \\ 
本科生,计算机系


\end{rSection}

%----------------------------------------------------------------------------------------
%	HONORS / AWARDS SECTION
%----------------------------------------------------------------------------------------



%----------------------------------------------------------------------------------------
%	PROJECTS / RESEARCH EXPERIENCE SECTION
%----------------------------------------------------------------------------------------

\begin{rSection}{项目 / 研究经历}

\begin{rSubsection}{x86模拟器} {\em 2019.10 - 2019.12}
{用C实现了一个基于qemu的x86模拟器。\\}
\item
\begin{itemize}
\setlength\itemsep{-0.5em}
\item[-] ALU. 定点和浮点数的运算
\item[-] 指令解码与执行, 内建调试器
\item[-] 内存管理, 实现了cache, 分段和分页机制
\item[-] 总代码量约2000行
\end{itemize}
\end{rSubsection}


\begin{rSubsection}{个人blog网站}{\em 2021.6 – 2021.8}{独立开发 \\}
{}
\item[]
\begin{itemize}
\setlength\itemsep{-0.5em}
\item[-] 学习web程序设计分析课程时自制的个人blog网站
\item[-] 登录系统和评论系统由Django框架开发
\item[-] 总代码量约1000行
\end{itemize}
\end{rSubsection}

\begin{rSubsection}{Python的API推荐}{\em 2021.12 – 2022.2}
{使用机器学习进行Python代码的API的智能推荐。\\} { }
\item[]
\begin{itemize}
\setlength\itemsep{-0.5em}
\item[-] 从Github收集了40个python项目
\item[-] 通过5种抽象语法单元推断data-flow, 给出API推荐列表
\item[-] 采用随机森林训练模型
\end{itemize}
\end{rSubsection}

\end{rSection}

%----------------------------------------------------------------------------------------
%	SKILLS SECTION
%----------------------------------------------------------------------------------------

\begin{rSection}{技能}
\begin{rSubsection}
{}{}{}{}
\item[-] 编程语言: 熟练使用 C/C++, Python, 了解shell, Matlab
%\item[-] 处理大数据: 熟悉Hadoop, MapReduce
\item[-] 机器学习:熟练使用numpy, matplotlib, sklearn
\end{rSubsection}
\end{rSection}

%----------------------------------------------------------------------------------------
%	OTHERS SECTION
%----------------------------------------------------------------------------------------

\begin{rSection}{其他}
\begin{rSubsection}
{}{}{}{}
\item[-] 数学水平: 离散数学, 数值分析掌握良好
\item[-] 英语水平: CET-6 566, TOEFL 99, 英语听说水平良好
\item[-] 初试成绩: 思想政治 75, 英语 87, 数学 121, 计算机学科专业基础 109. 总分392.
\end{rSubsection}
\end{rSection}

\end{document}
